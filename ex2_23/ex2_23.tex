%%%%%%%%%%%%%%%%%%%%%%%%%%%%%%%%%%%%%%%%%%%%%%%%%%%%%%%%%%%%%%%
%
% Welcome to writeLaTeX --- just edit your LaTeX on the left,
% and we'll compile it for you on the right. If you give
% someone the link to this page, they can edit at the same
% time. See the help menu above for more info. Enjoy!
%
%%%%%%%%%%%%%%%%%%%%%%%%%%%%%%%%%%%%%%%%%%%%%%%%%%%%%%%%%%%%%%%

% --------------------------------------------------------------
% This is all preamble stuff that you don't have to worry about.
% Head down to where it says "Start here"
% --------------------------------------------------------------
 
\documentclass[12pt]{article}
 
\usepackage[margin=1in]{geometry}
\usepackage{amsmath,amsthm,amssymb}
\usepackage{enumitem}
\usepackage{cancel}

\setlist[enumerate,1]{label={(\alph*)}} %this changes enumerate to (a),(b),...

\usepackage{graphicx} %package to manage images

\newcommand{\A}{{\mathcal{A}}}
\newcommand{\C}{{\mathbb C}}
\newcommand{\CC}{{\mathcal{C}}}
\newcommand{\N}{{\mathbb N}}
\newcommand{\R}{{\mathbb R}}
\newcommand{\Q}{{\mathbb Q}}
\newcommand{\Z}{{\mathbb Z}}

\newcommand{\Aut}{{\rm Aut}}
\newcommand{\End}{{\rm End}}
\newcommand{\Hom}{{\rm Hom}}
\newcommand{\id}{{\rm id}}
\newcommand{\Ima}{{\rm Im}}
\newcommand{\Ker}{{\rm Ker}}
\newcommand{\Mor}{{\rm Mor}}
\newcommand{\Rad}{{\rm Rad}}
\newcommand{\Prob}{{\sf P}}
\newcommand{\E}{{\sf E}}
\newcommand{\Var}{{\sf Var}}
\newcommand{\Cov}{{\sf Cov}}

\renewcommand\labelitemi{-} %this changes itemize bullet points to dashes (-)

\usepackage{listings}
\usepackage{xcolor}

%New colors defined below
\definecolor{codegreen}{rgb}{0,0.6,0}
\definecolor{codegray}{rgb}{0.5,0.5,0.5}
\definecolor{codepurple}{rgb}{0.58,0,0.82}
\definecolor{backcolour}{rgb}{0.95,0.95,0.92}

%Code listing style named "mystyle"
\lstdefinestyle{mystyle}{
  backgroundcolor=\color{backcolour}, commentstyle=\color{codegreen},
  keywordstyle=\color{magenta},
  numberstyle=\tiny\color{codegray},
  stringstyle=\color{codepurple},
  basicstyle=\ttfamily\footnotesize,
  breakatwhitespace=false,         
  breaklines=true,                 
  captionpos=b,                    
  keepspaces=true,                 
  numbers=left,                    
  numbersep=5pt,                  
  showspaces=false,                
  showstringspaces=false,
  showtabs=false,                  
  tabsize=2
}

%"mystyle" code listing set
\lstset{style=mystyle}
 
\newenvironment{theorem}[2][Theorem]{\begin{trivlist}
\item[\hskip \labelsep {\bfseries #1}\hskip \labelsep {\bfseries #2.}]}
{\end{trivlist}}
\newenvironment{lemma}[2][Lemma]{\begin{trivlist}
\item[\hskip \labelsep {\bfseries #1}\hskip \labelsep {\bfseries #2.}]}
{\end{trivlist}}
\newenvironment{exercise}[2][Exercise]{\begin{trivlist}
\item[\hskip \labelsep {\bfseries #1}\hskip \labelsep {\bfseries #2.}]}
{\end{trivlist}}
\newenvironment{problem}[2][Problem]{\begin{trivlist}
\item[\hskip \labelsep {\bfseries #1}\hskip \labelsep {\bfseries #2.}]}
{\end{trivlist}}
\newenvironment{question}[2][Question]{\begin{trivlist}
\item[\hskip \labelsep {\bfseries #1}\hskip \labelsep {\bfseries #2.}]}
{\end{trivlist}}
\newenvironment{corollary}[2][Corollary]{\begin{trivlist}
\item[\hskip \labelsep {\bfseries #1}\hskip \labelsep {\bfseries #2.}]}
{\end{trivlist}}

\newenvironment{solution}{\begin{proof}[Solution]}{\end{proof}}
 
\begin{document}
 
% --------------------------------------------------------------
%                         Start here
% --------------------------------------------------------------
 
\title{Exam 2 2023}%replace X with the appropriate number
\author{Mengxiang Jiang\\ %replace with your name
Stat 610 Distribution Theory} %if necessary, replace with your course title
 
\maketitle
 
\begin{problem}{1} %You can use theorem, exercise, problem, or question here.
  $X$ and $Y$ are independent random variables with moment generating functions
  $M_X (t) = \frac{e^t}{1 - t^2}$ and $M_Y (t) = e^{e^t - t - 1}$, respectively.
  Find $\Var(X + Y)$.
  \\\\
  \begin{align*}
    \Var(X + Y) &= \Var(X) + \Var(Y) \\
    &= \frac{d^2\ln(M_X (t))}{dt^2} \Big|_{t=0} + 
    \frac{d^2\ln(M_Y (t))}{dt^2} \Big|_{t=0} \\
    &= \frac{d^2}{dt^2} \left( t - \ln(1 - t^2) \right) \Big|_{t=0} + 
    \frac{d^2}{dt^2} \left( e^t - t - 1 \right) \Big|_{t=0} \\
    &= \frac{d}{dt} \left( 1 + \frac{2t}{1 - t^2} \right) \Big|_{t=0} + 
    \frac{d}{dt} \left( e^t - 1 \right) \Big|_{t=0} \\
    &= \left( 2(1-t^2)^{-2} + 2t(-2)(-2t)(1-t^2)^{-3} \right) \Big|_{t=0} + 
    e^t \Big|_{t=0} \\
    &= 2 + 1 = 3
  \end{align*}
\end{problem}

\begin{problem}{2} 
  $(S, T)$ has joint pdf $f_{S,T} (s, t) = \frac{(s+t)^2}{6} e^{-s-t}$ for $s > 0, t > 0$.
  \begin{enumerate}
    \item Let $W = S + T$ and $Z = \frac{S}{S+T}$. Find the joint pdf for 
    $(W, Z)$ and identify.
    \item Accept as given that $E(S) = E(T) = 2$. Find $\Cov(S, T)$.
  \end{enumerate}
  \begin{enumerate}
    \item We have the transformation
    \[  \begin{cases}
      W = S + T \\
      Z = \frac{S}{S + T}
    \end{cases} \]
    The inverse transformation is
    \[  \begin{cases}
      S = WZ \\
      T = W(1-Z)
    \end{cases} \]
    The Jacobian determinant is
    \[
      J = \begin{vmatrix}
        \frac{\partial S}{\partial W} & \frac{\partial S}{\partial Z} \\
        \frac{\partial T}{\partial W} & \frac{\partial T}{\partial Z}
      \end{vmatrix} =
      \begin{vmatrix}
        Z & W \\
        1-Z & -W
      \end{vmatrix} = -WZ - W(1-Z) = -W
    \]
    Thus, the joint pdf for $(W, Z)$ is
    \begin{align*}
      f_{W,Z} (w, z) &= f_{S,T} (wz, w(1-z)) \cdot |J| \\
      &= \frac{(wz + w(1-z))^2}{6} e^{-wz - w(1-z)} \cdot w \\
      &= \frac{w^2}{6} e^{-w} \cdot w = \frac{w^3}{6} e^{-w}
    \end{align*}
    for $w > 0$ and $0 < z < 1$. This shows that $W$ and $Z$ are independent, with
    $W \sim \text{gamma}(4, 1)$ and $Z \sim \text{uniform}(0, 1)$.
    \item We have
    \begin{align*}
      \Cov(S, T) &= \E(ST) - \E(S)\E(T) \\
      &= \E(W^2 Z(1-Z)) - 4 \\
      &= \E(W^2) \E(Z(1-Z)) - 4 \\
      &= \left( \Var(W) + (\E(W))^2 \right) \left( \E(Z) - \E(Z^2) \right) - 4 \\
      &= (4 + 16) \left( \frac{1}{2} - \frac{1}{3} \right) - 4 \\
      &= 20 \cdot \frac{1}{6} - 4 = \frac{10}{3} - 4 = -\frac{2}{3}
    \end{align*}
  \end{enumerate}
\end{problem}

\begin{problem}{3}
  Suppose X and Y are positive integer-valued random variables with joint pmf
  \[
    f_{X,Y} (x, y) =  C 1_{\{1,...,10\}}(x) 1_{\{1,...,x\}}(y)
  \]
  for some constant $C$.\\
  What is the best predictor of $Y$ as a function of $X$? Explain.
  \\\\
  Normalize the pmf:
  \begin{align*}
    1 &= \sum_{x=1}^{10} \sum_{y=1}^{x} C \\
    &= C \sum_{x=1}^{10} x = C \cdot \frac{10 \cdot 11}{2} = 55C \\
    \Rightarrow C &= \frac{1}{55}
  \end{align*}
  Then the conditional pmf of $Y$ given $X$ is
  \[
    f_{Y|X} (y|x) = \frac{f_{X,Y} (x,y)}{f_X (x)} = 
    \frac{\frac{1}{55}}{\sum_{y=1}^{x} \frac{1}{55}} = \frac{1}{x}
  \]
  for $y = 1, ..., x$. Thus,
  \[
    \E(Y|X=x) = \sum_{y=1}^{x} y \cdot \frac{1}{x} = \frac{x(x+1)}{2x} 
    = \frac{x+1}{2}
  \]
  This is the best predictor of $Y$ as a function of $X$ 
  since it minimizes the mean squared error by the orthogonality principle.
\end{problem}

\begin{problem}{4}
  Show that the beta distributions form a two-parameter exponential family.
  \\\\
  The pdf of a beta distribution is
  \[
    f_X (x) = \frac{\Gamma(\alpha + \beta)}{\Gamma(\alpha) \Gamma(\beta)}
    x^{\alpha - 1} (1 - x)^{\beta - 1}
  \]
  for $0 < x < 1$, $\alpha > 0$, and $\beta > 0$. We can rewrite it as
  \begin{align*}
    f_X (x) &= \exp \left( \ln \left( \frac{\Gamma(\alpha + \beta)}
    {\Gamma(\alpha) \Gamma(\beta)}
    x^{\alpha - 1} (1 - x)^{\beta - 1} \right) \right) \\
    &= \exp \left( \ln \left( \frac{\Gamma(\alpha + \beta)}
    {\Gamma(\alpha) \Gamma(\beta)} \right)
    + (\alpha - 1) \ln(x) + (\beta - 1) \ln(1 - x) \right) \\
    &= \exp \left( (\alpha - 1) \ln(x) + (\beta - 1) \ln(1 - x)
    + \ln \left( \frac{\Gamma(\alpha + \beta)}
    {\Gamma(\alpha) \Gamma(\beta)} \right) \right)
  \end{align*}
  This is in the form of a two-parameter exponential family with
  \[
    f(x) = c(\theta)h(x)e^{w_1(\theta) t_1(x) + w_2(\theta) t_2(x)}
  \]
  where
  \[
    c(\theta) = \frac{\Gamma(\alpha + \beta)}{\Gamma(\alpha) \Gamma(\beta)}, \quad
    h(x) = 1, \quad
    w_1(\theta) = \alpha - 1, \quad
    w_2(\theta) = \beta - 1, \quad
    t_1(x) = \ln(x), \quad
    t_2(x) = \ln(1 - x)
  \]
\end{problem}

% --------------------------------------------------------------
%     You don't have to mess with anything below this line.
% --------------------------------------------------------------
\end{document}