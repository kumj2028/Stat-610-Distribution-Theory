%%%%%%%%%%%%%%%%%%%%%%%%%%%%%%%%%%%%%%%%%%%%%%%%%%%%%%%%%%%%%%%
%
% Welcome to writeLaTeX --- just edit your LaTeX on the left,
% and we'll compile it for you on the right. If you give
% someone the link to this page, they can edit at the same
% time. See the help menu above for more info. Enjoy!
%
%%%%%%%%%%%%%%%%%%%%%%%%%%%%%%%%%%%%%%%%%%%%%%%%%%%%%%%%%%%%%%%

% --------------------------------------------------------------
% This is all preamble stuff that you don't have to worry about.
% Head down to where it says "Start here"
% --------------------------------------------------------------
 
\documentclass[12pt]{article}
 
\usepackage[margin=1in]{geometry}
\usepackage{amsmath,amsthm,amssymb}
\usepackage{enumitem}
\usepackage{cancel}

\setlist[enumerate,1]{label={(\alph*)}} %this changes enumerate to (a),(b),...

\usepackage{graphicx} %package to manage images

\newcommand{\A}{{\mathcal{A}}}
\newcommand{\C}{{\mathbb C}}
\newcommand{\CC}{{\mathcal{C}}}
\newcommand{\N}{{\mathbb N}}
\newcommand{\R}{{\mathbb R}}
\newcommand{\Q}{{\mathbb Q}}
\newcommand{\Z}{{\mathbb Z}}

\newcommand{\Aut}{{\rm Aut}}
\newcommand{\End}{{\rm End}}
\newcommand{\Hom}{{\rm Hom}}
\newcommand{\id}{{\rm id}}
\newcommand{\Ima}{{\rm Im}}
\newcommand{\Ker}{{\rm Ker}}
\newcommand{\Mor}{{\rm Mor}}
\newcommand{\Rad}{{\rm Rad}}
\newcommand{\Prob}{{\sf P}}
\newcommand{\E}{{\sf E}}
\newcommand{\Var}{{\sf Var}}

\renewcommand\labelitemi{-} %this changes itemize bullet points to dashes (-)

\usepackage{listings}
\usepackage{xcolor}

%New colors defined below
\definecolor{codegreen}{rgb}{0,0.6,0}
\definecolor{codegray}{rgb}{0.5,0.5,0.5}
\definecolor{codepurple}{rgb}{0.58,0,0.82}
\definecolor{backcolour}{rgb}{0.95,0.95,0.92}

%Code listing style named "mystyle"
\lstdefinestyle{mystyle}{
  backgroundcolor=\color{backcolour}, commentstyle=\color{codegreen},
  keywordstyle=\color{magenta},
  numberstyle=\tiny\color{codegray},
  stringstyle=\color{codepurple},
  basicstyle=\ttfamily\footnotesize,
  breakatwhitespace=false,         
  breaklines=true,                 
  captionpos=b,                    
  keepspaces=true,                 
  numbers=left,                    
  numbersep=5pt,                  
  showspaces=false,                
  showstringspaces=false,
  showtabs=false,                  
  tabsize=2
}

%"mystyle" code listing set
\lstset{style=mystyle}
 
\newenvironment{theorem}[2][Theorem]{\begin{trivlist}
\item[\hskip \labelsep {\bfseries #1}\hskip \labelsep {\bfseries #2.}]}
{\end{trivlist}}
\newenvironment{lemma}[2][Lemma]{\begin{trivlist}
\item[\hskip \labelsep {\bfseries #1}\hskip \labelsep {\bfseries #2.}]}
{\end{trivlist}}
\newenvironment{exercise}[2][Exercise]{\begin{trivlist}
\item[\hskip \labelsep {\bfseries #1}\hskip \labelsep {\bfseries #2.}]}
{\end{trivlist}}
\newenvironment{problem}[2][Problem]{\begin{trivlist}
\item[\hskip \labelsep {\bfseries #1}\hskip \labelsep {\bfseries #2.}]}
{\end{trivlist}}
\newenvironment{question}[2][Question]{\begin{trivlist}
\item[\hskip \labelsep {\bfseries #1}\hskip \labelsep {\bfseries #2.}]}
{\end{trivlist}}
\newenvironment{corollary}[2][Corollary]{\begin{trivlist}
\item[\hskip \labelsep {\bfseries #1}\hskip \labelsep {\bfseries #2.}]}
{\end{trivlist}}

\newenvironment{solution}{\begin{proof}[Solution]}{\end{proof}}
 
\begin{document}
 
% --------------------------------------------------------------
%                         Start here
% --------------------------------------------------------------
 
\title{Exam 1 2023}%replace X with the appropriate number
\author{Mengxiang Jiang\\ %replace with your name
Stat 610 Distribution Theory} %if necessary, replace with your course title
 
\maketitle
 
\begin{problem}{1} %You can use theorem, exercise, problem, or question here.
  Consider the following continuous cdf for a random variable $W$.
  \[
    F_W (x) =
    \begin{cases}
      0 & \text{if } x \le 0, \\
      \frac{x}{3} & \text{if } 0 < x < 1, \\
      \frac{1}{3} & \text{if } 1 \le x < 2, \\
      \frac{2x}{3} - 1 & \text{if } 2 \le x < 3, \\
      1 & \text{if } 3 \le x.
    \end{cases}
  \]
  Find the probability density function (pdf) and $\E(W)$.
  \\\\
  Since $F_W (x)$ is continuous, we can take the derivative to get the pdf.
  \[
    f_W (x) =
    \begin{cases}
      \frac{1}{3} & \text{if } 0 < x < 1, \\
      0 & \text{if } 1 \le x < 2, \\
      \frac{2}{3} & \text{if } 2 \le x < 3, \\
      0 & \text{otherwise}.
    \end{cases}
  \]
  Then we can calculate $\E(W)$ as follows.
  \[
    \begin{aligned}
      \E(W) &= \int_{-\infty}^{\infty} x f_W (x) dx \\
      &= \int_0^1 x \cdot \frac{1}{3} dx 
      + \int_1^2 x \cdot 0 dx + \int_2^3 x \cdot \frac{2}{3} dx \\
      &= \left. \frac{x^2}{6} \right|_0^1 + 0 
      + \left. \frac{x^2}{3} \right|_2^3 \\
      &= \frac{1}{6} + 0 + (3 - \frac{4}{3}) = \frac{11}{6}.
    \end{aligned}
  \]
\end{problem}

\begin{problem}{2}
  A population of 1000 school districts consists of 150 districts with 
  1 high school, 350 with 2 high schools, 300 with 3 high schools, 
  and 200 with 4 high schools. A researcher selects two districts at random, 
  without replacement. Given that the two districts
  have the same number of high schools, what is the chance 
  they have $x$ high schools, for each
  $x = 1, 2, 3, 4$? Computable expressions suffice.
  \\\\
  Let $A$ be the event that the two selected districts have the same number of high schools.
  Let $B_x$ be the event that the two selected districts have $x$ high schools.
  We want to find $\Prob(B_x | A)$ for $x = 1, 2, 3, 4$.
  By the definition of conditional probability, we have
  \[
    \Prob(B_x | A) = \frac{\Prob(B_x \cap A)}{\Prob(A)} 
    = \frac{\Prob(B_x)}{\Prob(A)},
  \]
  since $B_x \subseteq A$.
  We can calculate $\Prob(A)$ as follows.
  \[
    \begin{aligned}
      \Prob(A) &= \Prob(B_1) + \Prob(B_2) + \Prob(B_3) + \Prob(B_4) \\
      &= \frac{\binom{150}{2}}{\binom{1000}{2}} 
      + \frac{\binom{350}{2}}{\binom{1000}{2}} 
      + \frac{\binom{300}{2}}{\binom{1000}{2}} 
      + \frac{\binom{200}{2}}{\binom{1000}{2}} \\
      &= \frac{\binom{150}{2} + \binom{350}{2} 
      + \binom{300}{2} + \binom{200}{2}}{\binom{1000}{2}}.
    \end{aligned}
  \]
  Then we can calculate $\Prob(B_x | A)$ as follows.
  \[
    \Prob(B_x | A) = \frac{\Prob(B_x)}{\Prob(A)} 
    = \frac{\frac{\binom{n_x}{2}}{\binom{1000}{2}}}
    {\frac{\binom{150}{2} + \binom{350}{2} + \binom{300}{2} 
    + \binom{200}{2}}{\binom{1000}{2}}}
    = \frac{\binom{n_x}{2}}{\binom{150}{2} + \binom{350}{2} + \binom{300}{2} + \binom{200}{2}},
  \]
  where $n_x$ is the number of districts with $x$ high schools.
\end{problem}

\begin{problem}{3}
Suppose $T$ has gamma(2,1) distribution.
\begin{enumerate}
  \item Determine the cumulative distribution function (cdf), 
  for all real values, and provide an equation that the 
  median must solve. (Do not try to solve it.)
  \item Find the pdf for $Y = T^{1/3}$.
\end{enumerate} 

\begin{enumerate}
  \item Since the pdf of $T$ is given by
  \[
    f_T (t) =
    \begin{cases}
      t e^{-t} & \text{if } t > 0, \\
      0 & \text{otherwise},
    \end{cases}
  \]
  we can calculate the cdf of $T$ as follows.
  \[
    \begin{aligned}
      F_T (t) &= \int_{-\infty}^{t} f_T (x) dx \\
      &= \int_0^t x e^{-x} dx \quad (\text{since } f_T (x) = 0 \text{ for } x \le 0)\\
      &= \left. -xe^{-x} - e^{-x} \right|_0^t \quad (\text{by integration by parts})\\
      &= -te^{-t} - e^{-t} + 1 \\
      &= 1 - e^{-t} - te^{-t}.
    \end{aligned}
  \]
  Thus the cdf of $T$ is given by
  \[
    F_T (t) =
    \begin{cases}
      0 & \text{if } t \le 0, \\
      1 - e^{-t} - te^{-t} & \text{if } t > 0.
    \end{cases}
  \]
  The median must satisfy the equation
  \[
    F_T (m) = \frac{1}{2}.
  \]
  \item The pdf of $T$ is given by
  \[
    f_T (t) =
    \begin{cases}
      t e^{-t} & \text{if } t > 0, \\
      0 & \text{otherwise}.
    \end{cases}
  \]
  Since $Y = T^{1/3}$, we have $T = Y^3$ and $\frac{dT}{dY} = 3Y^2$.
  Then we can calculate the pdf of $Y$ as follows.
  \[
    f_Y (y) = f_T (y^3) \left| \frac{dT}{dY} \right|
    = f_T (y^3) \cdot 3y^2 =
    \begin{cases}
      3y^5 e^{-y^3} & \text{if } y > 0, \\
      0 & \text{otherwise}.
    \end{cases}
  \]
\end{enumerate}

\end{problem}

\begin{problem}{4}
  Derive the moment generating function (mgf) for the geometric($p$) pmf. 
  (Recall that $\sum_{k=1}^{\infty} a^k = \frac{a}{1 - a}$ 
  for $|a| < 1$.)
  \\\\
  The pmf of a geometric($p$) random variable $X$ is given by
  \[
    f_X (x) =
    \begin{cases}
      p(1-p)^{x-1} & \text{if } x = 1, 2, \ldots, \\
      0 & \text{otherwise}.
    \end{cases}
  \]
  Then we can calculate the mgf of $X$ as follows.
  \[
    \begin{aligned}
      M_X (t) &= \E(e^{tX}) \\
      &= \sum_{x=1}^{\infty} e^{tx} f_X (x) \\
      &= \sum_{x=1}^{\infty} e^{tx} p(1-p)^{x-1} \\
      &= \frac{p}{1-p} \sum_{x=1}^{\infty} \left[e^t (1-p)\right]^x \\
      &= \frac{p}{1-p} \cdot \frac{e^t (1-p)}{1 - e^t (1-p)} \quad
      (\text{since } |e^t (1-p)| < 1) \\
      &= \frac{p e^t}{1 - (1-p)e^t}.
    \end{aligned}
  \]

\end{problem}

% --------------------------------------------------------------
%     You don't have to mess with anything below this line.
% --------------------------------------------------------------
\end{document}