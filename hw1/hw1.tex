%%%%%%%%%%%%%%%%%%%%%%%%%%%%%%%%%%%%%%%%%%%%%%%%%%%%%%%%%%%%%%%
%
% Welcome to writeLaTeX --- just edit your LaTeX on the left,
% and we'll compile it for you on the right. If you give
% someone the link to this page, they can edit at the same
% time. See the help menu above for more info. Enjoy!
%
%%%%%%%%%%%%%%%%%%%%%%%%%%%%%%%%%%%%%%%%%%%%%%%%%%%%%%%%%%%%%%%

% --------------------------------------------------------------
% This is all preamble stuff that you don't have to worry about.
% Head down to where it says "Start here"
% --------------------------------------------------------------
 
\documentclass[12pt]{article}
 
\usepackage[margin=1in]{geometry}
\usepackage{amsmath,amsthm,amssymb}
\usepackage{enumitem}

\setlist[enumerate,1]{label={(\alph*)}} %this changes enumerate to (a),(b),...

\usepackage{graphicx} %package to manage images

\newcommand{\A}{{\mathcal{A}}}
\newcommand{\C}{{\mathbb C}}
\newcommand{\CC}{{\mathcal{C}}}
\newcommand{\N}{{\mathbb N}}
\newcommand{\R}{{\mathbb R}}
\newcommand{\Q}{{\mathbb Q}}
\newcommand{\Z}{{\mathbb Z}}

\newcommand{\Aut}{{\rm Aut}}
\newcommand{\End}{{\rm End}}
\newcommand{\Hom}{{\rm Hom}}
\newcommand{\id}{{\rm id}}
\newcommand{\Ima}{{\rm Im}}
\newcommand{\Ker}{{\rm Ker}}
\newcommand{\Mor}{{\rm Mor}}
\newcommand{\Rad}{{\rm Rad}}

\renewcommand\labelitemi{-} %this changes itemize bullet points to dashes (-)

\usepackage{listings}
\usepackage{xcolor}

%New colors defined below
\definecolor{codegreen}{rgb}{0,0.6,0}
\definecolor{codegray}{rgb}{0.5,0.5,0.5}
\definecolor{codepurple}{rgb}{0.58,0,0.82}
\definecolor{backcolour}{rgb}{0.95,0.95,0.92}

%Code listing style named "mystyle"
\lstdefinestyle{mystyle}{
  backgroundcolor=\color{backcolour}, commentstyle=\color{codegreen},
  keywordstyle=\color{magenta},
  numberstyle=\tiny\color{codegray},
  stringstyle=\color{codepurple},
  basicstyle=\ttfamily\footnotesize,
  breakatwhitespace=false,         
  breaklines=true,                 
  captionpos=b,                    
  keepspaces=true,                 
  numbers=left,                    
  numbersep=5pt,                  
  showspaces=false,                
  showstringspaces=false,
  showtabs=false,                  
  tabsize=2
}

%"mystyle" code listing set
\lstset{style=mystyle}
 
\newenvironment{theorem}[2][Theorem]{\begin{trivlist}
\item[\hskip \labelsep {\bfseries #1}\hskip \labelsep {\bfseries #2.}]}
{\end{trivlist}}
\newenvironment{lemma}[2][Lemma]{\begin{trivlist}
\item[\hskip \labelsep {\bfseries #1}\hskip \labelsep {\bfseries #2.}]}
{\end{trivlist}}
\newenvironment{exercise}[2][Exercise]{\begin{trivlist}
\item[\hskip \labelsep {\bfseries #1}\hskip \labelsep {\bfseries #2.}]}
{\end{trivlist}}
\newenvironment{problem}[2][Problem]{\begin{trivlist}
\item[\hskip \labelsep {\bfseries #1}\hskip \labelsep {\bfseries #2.}]}
{\end{trivlist}}
\newenvironment{question}[2][Question]{\begin{trivlist}
\item[\hskip \labelsep {\bfseries #1}\hskip \labelsep {\bfseries #2.}]}
{\end{trivlist}}
\newenvironment{corollary}[2][Corollary]{\begin{trivlist}
\item[\hskip \labelsep {\bfseries #1}\hskip \labelsep {\bfseries #2.}]}
{\end{trivlist}}

\newenvironment{solution}{\begin{proof}[Solution]}{\end{proof}}
 
\begin{document}
 
% --------------------------------------------------------------
%                         Start here
% --------------------------------------------------------------
 
\title{Homework 1}%replace X with the appropriate number
\author{Mengxiang Jiang\\ %replace with your name
Stat 610 Distribution Theory} %if necessary, replace with your course title
 
\maketitle
 
\begin{problem}{1} %You can use theorem, exercise, problem, or question here.
  \textit{Statistical Inference} by Casella and Berger, 2nd Edition, Chapter 1, 
  Exercise 4, 5, and 6.
  \begin{itemize}
    \item[4] For events $A$ and $B$, find formulas for the probabilities of the 
    following events in terms of the quantities $P(A)$, $P(B)$, and $P(A \cap B)$.
      \begin{itemize}
        \item[(a)] either $A$ or $B$ or both
        \item[(b)] either $A$ or $B$ but not both
        \item[(c)] at least one of $A$ or $B$
        \item[(d)] at most one of $A$ or $B$
      \end{itemize}
      \begin{itemize}
        \item[(a)] $P(A \cup B) = P(A) + P(B) - P(A \cap B)$
        \item[(b)] $P(A \cup B) - P(A \cap B) = P(A) + P(B) - 2P(A \cap B)$
        \item[(c)] $P(A \cup B) = P(A) + P(B) - P(A \cap B)$
        \item[(d)] $1 - P(A \cap B)$
      \end{itemize}
    \item[5] Approximately one-third of all human twins are identical 
    (one-egg) and two-thirds are fraternal (two-egg) twins. Identical twins 
    are necessarily the same sex, with male and female being equally likely.
    Among fraternal twins, approximately one-fourth are both female, one-fourth
    are both male, and half are one male and one female. Finally, among all
    U.S. births, approximately 1 in 90 is a twin birth. Define the following
    events:
      \begin{itemize}
        \item $A$ = {the birth results in twin females}
        \item $B$ = {the twins are identical twins}
        \item $C$ = {a U.S. birth results in twins}
      \end{itemize}
      \begin{itemize}
        \item[(a)] State, in words, the event $A \cap B \cap C$.
        \item[(b)] Find $P(A \cap B \cap C)$.
      \end{itemize}
      \begin{itemize}
        \item[(a)] The event that a U.S. birth results in identical
        twin females.
        \item[(b)] From the given information, we have
        \begin{itemize}
          \item $P(C) = \frac{1}{90}$
          \item $P(B|C) = \frac{1}{3}$
          \item $P(A|B,C) = \frac{1}{2}$
        \end{itemize}
        So, $P(A \cap B \cap C) = P(C)P(B|C)P(A|B,C) = \frac{1}{90} \times
        \frac{1}{3} \times \frac{1}{2} = \frac{1}{540}$.
      \end{itemize} 
    \item[6] Two pennies, one with $P(\text{head}) = u$ and one with
    $P(\text{head}) = w$, are to be tossed together independently. Define
    \begin{itemize}
      \item $p_0 = P(\text{0 heads occur})$,
      \item $p_1 = P(\text{1 head occurs})$,
      \item $p_2 = P(\text{2 heads occur})$.
    \end{itemize}
    Can $u$ and $w$ be chosen so that $p_0 = p_1 = p_2$? Prove your answer.
    \\\\
    We have
    \begin{itemize}
      \item $p_0 = (1-u)(1-w)$,
      \item $p_1 = u(1-w) + w(1-u)$,
      \item $p_2 = uw$.
    \end{itemize}
    So, $p_0 = p_2$ implies $(1-u)(1-w) = uw$, which simplifies to
    $u + w = 1$. Also, $p_1 = p_2$ implies $u(1-w) + w(1-u) = uw$, which
    simplifies to $ u + w = 3uw$. Combining these two equations, we have
    $3uw = 1$, or $uw = \frac{1}{3}$. However, since $u + w = 1$, by AM-GM
    inequality, we have $\frac{u+w}{2} = \frac{1}{2} \ge \sqrt{uw}$, which
    when squared gives $\frac{1}{4} \ge uw$. This
    contradicts $uw = \frac{1}{3}$. Therefore, there are no such $u$ and $w$,
    assuming they are both in $[0,1]$.
  \end{itemize}
\end{problem}

\begin{problem}{2}
  Here is a sample space: $\mathcal{S} = \{a, b, c\}$.
  \begin{enumerate}
    \item Explicitly provide the $\sigma$-algebra of all subsets.
    \item Suppose one may only observe whether the outcome is $a$ or not.
    Explicitly provide the smallest relevant $\sigma$-algebra. Hint: what
    events may be obtained by complements, unions and intersections,
    starting only with ${a}$?
  \end{enumerate}
\end{problem}

% --------------------------------------------------------------
%     You don't have to mess with anything below this line.
% --------------------------------------------------------------
\end{document}