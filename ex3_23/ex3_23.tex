%%%%%%%%%%%%%%%%%%%%%%%%%%%%%%%%%%%%%%%%%%%%%%%%%%%%%%%%%%%%%%%
%
% Welcome to writeLaTeX --- just edit your LaTeX on the left,
% and we'll compile it for you on the right. If you give
% someone the link to this page, they can edit at the same
% time. See the help menu above for more info. Enjoy!
%
%%%%%%%%%%%%%%%%%%%%%%%%%%%%%%%%%%%%%%%%%%%%%%%%%%%%%%%%%%%%%%%

% --------------------------------------------------------------
% This is all preamble stuff that you don't have to worry about.
% Head down to where it says "Start here"
% --------------------------------------------------------------
 
\documentclass[12pt]{article}
 
\usepackage[margin=1in]{geometry}
\usepackage{amsmath,amsthm,amssymb}
\usepackage{enumitem}
\usepackage{cancel}

\setlist[enumerate,1]{label={(\alph*)}} %this changes enumerate to (a),(b),...

\usepackage{graphicx} %package to manage images

\newcommand{\A}{{\mathcal{A}}}
\newcommand{\C}{{\mathbb C}}
\newcommand{\CC}{{\mathcal{C}}}
\newcommand{\N}{{\mathbb N}}
\newcommand{\R}{{\mathbb R}}
\newcommand{\Q}{{\mathbb Q}}
\newcommand{\Z}{{\mathbb Z}}

\newcommand{\Aut}{{\rm Aut}}
\newcommand{\End}{{\rm End}}
\newcommand{\Hom}{{\rm Hom}}
\newcommand{\id}{{\rm id}}
\newcommand{\Ima}{{\rm Im}}
\newcommand{\Ker}{{\rm Ker}}
\newcommand{\Mor}{{\rm Mor}}
\newcommand{\Rad}{{\rm Rad}}
\newcommand{\Prob}{{\sf P}}
\newcommand{\E}{{\sf E}}
\newcommand{\Var}{{\sf Var}}
\newcommand{\Cov}{{\sf Cov}}

\renewcommand\labelitemi{-} %this changes itemize bullet points to dashes (-)

\usepackage{listings}
\usepackage{xcolor}

%New colors defined below
\definecolor{codegreen}{rgb}{0,0.6,0}
\definecolor{codegray}{rgb}{0.5,0.5,0.5}
\definecolor{codepurple}{rgb}{0.58,0,0.82}
\definecolor{backcolour}{rgb}{0.95,0.95,0.92}

%Code listing style named "mystyle"
\lstdefinestyle{mystyle}{
  backgroundcolor=\color{backcolour}, commentstyle=\color{codegreen},
  keywordstyle=\color{magenta},
  numberstyle=\tiny\color{codegray},
  stringstyle=\color{codepurple},
  basicstyle=\ttfamily\footnotesize,
  breakatwhitespace=false,         
  breaklines=true,                 
  captionpos=b,                    
  keepspaces=true,                 
  numbers=left,                    
  numbersep=5pt,                  
  showspaces=false,                
  showstringspaces=false,
  showtabs=false,                  
  tabsize=2
}

%"mystyle" code listing set
\lstset{style=mystyle}
 
\newenvironment{theorem}[2][Theorem]{\begin{trivlist}
\item[\hskip \labelsep {\bfseries #1}\hskip \labelsep {\bfseries #2.}]}
{\end{trivlist}}
\newenvironment{lemma}[2][Lemma]{\begin{trivlist}
\item[\hskip \labelsep {\bfseries #1}\hskip \labelsep {\bfseries #2.}]}
{\end{trivlist}}
\newenvironment{exercise}[2][Exercise]{\begin{trivlist}
\item[\hskip \labelsep {\bfseries #1}\hskip \labelsep {\bfseries #2.}]}
{\end{trivlist}}
\newenvironment{problem}[2][Problem]{\begin{trivlist}
\item[\hskip \labelsep {\bfseries #1}\hskip \labelsep {\bfseries #2.}]}
{\end{trivlist}}
\newenvironment{question}[2][Question]{\begin{trivlist}
\item[\hskip \labelsep {\bfseries #1}\hskip \labelsep {\bfseries #2.}]}
{\end{trivlist}}
\newenvironment{corollary}[2][Corollary]{\begin{trivlist}
\item[\hskip \labelsep {\bfseries #1}\hskip \labelsep {\bfseries #2.}]}
{\end{trivlist}}

\newenvironment{solution}{\begin{proof}[Solution]}{\end{proof}}
 
\begin{document}
 
% --------------------------------------------------------------
%                         Start here
% --------------------------------------------------------------
 
\title{Final Exam 2023}%replace X with the appropriate number
\author{Mengxiang Jiang\\ %replace with your name
Stat 610 Distribution Theory} %if necessary, replace with your course title
 
\maketitle
 
\begin{problem}{1} %You can use theorem, exercise, problem, or question here.
Suppose $X_1, \ldots, X_n$ are iid with pdf 
$f_X (x) = \frac{2x^3}{\beta^4} e^{-(x/\beta)^2}$
for $x > 0$. This has moments
\[
  \E(X) = \frac{3\sqrt{\pi}}{4} \beta, \quad \E(X^2) = 2\beta^2, 
  \quad \E(X^3) = \frac{15\sqrt{\pi}}{8} \beta^3, \quad \E(X^4) = 6\beta^4.
\]
\begin{enumerate}
  \item Determine $C$ and $V$ such that 
  \[
    \sqrt{n} \left( \frac{1}{2n} \sum_{i=1}^n X_i^2 - C \right) 
    \xrightarrow{D} \text{normal}(0, V)
  \] 
  as $n \to \infty$, with explanation.
  \item Find the pdf for $Y = (X/\beta)^2$ and identify the distribution.
\end{enumerate}
\begin{enumerate}
  \item By the Weak Law of Large Numbers (WLLN), we have
  \[
    \frac{1}{n} \sum_{i=1}^n X_i^2 \xrightarrow{P} \E(X^2) = 2\beta^2.
  \]
  Thus, we can choose $C = \beta^2$.
  By the Central Limit Theorem (CLT), we have
  \[
    \sqrt{n} \left( \frac{1}{n} \sum_{i=1}^n X_i^2 - \E(X^2) \right) 
    \xrightarrow{D} \text{normal}(0, \Var(X^2)).
  \]
  Therefore,
  \begin{align*}
    \sqrt{n} \left( \frac{1}{2n} \sum_{i=1}^n X_i^2 - C \right) 
    &= \frac{1}{2} \sqrt{n} \left( \frac{1}{n} \sum_{i=1}^n X_i^2 - \E(X^2) \right) \\
    &\xrightarrow{D} \text{normal}\left( 0, \frac{1}{4} \Var(X^2) \right).
  \end{align*}
  Next, we calculate $\Var(X^2)$:
  \begin{align*}
    \Var(X^2) &= \E(X^4) - (\E(X^2))^2 \\
    &= 6\beta^4 - (2\beta^2)^2 \\
    &= 2\beta^4.
  \end{align*}
  Thus, we can choose $V = \frac{1}{4} \Var(X^2) = \frac{1}{4} \cdot 2\beta^4 = \frac{1}{2} \beta^4$.
  \item To find the pdf of $Y = (X/\beta)^2$, we use the transformation method.
  The inverse transformation is $X = \beta \sqrt{Y}$. The Jacobian is
  \[
    J = \left| \frac{dX}{dY} \right| = \left| \frac{\beta}{2\sqrt{Y}} \right| = \frac{\beta}{2\sqrt{Y}}.
  \]
  The pdf of $Y$ is given by
  \begin{align*}
    f_Y (y) &= f_X (\beta \sqrt{y}) \cdot J \\
    &= \frac{2(\beta \sqrt{y})^3}{\beta^4} e^{-(\beta \sqrt{y}/\beta)^2} \cdot \frac{\beta}{2\sqrt{y}} \\
    &= \frac{2\beta^3 y^{3/2}}{\beta^4} e^{-y} \cdot \frac{\beta}{2\sqrt{y}} \\
    &= y e^{-y}, \quad y > 0.
  \end{align*}
  This is the pdf of a Gamma distribution with shape parameter $k = 2$ and scale parameter $\theta = 1$.
\end{enumerate}
\end{problem}

\begin{problem}{2}
Assume $Y_1, \ldots, Y_n$ are iid with cdf $F_Y (y) = 1 - (1 + y)e^{-y}$.
\begin{enumerate}
  \item Define $C_1 = \sum_{i=1}^n 1_{[0,1]}(Y_i)$, 
  $C_2 = \sum_{i=1}^n 1_{(1,3]}(Y_i)$ and 
  $C_3 = \sum_{i=1}^n 1_{(3,\infty)}(Y_i)$, so that
  $(C_1, C_2, C_3)$ has trinomial distribution. 
  What is the mean and variance of each $C_i$?
  \item Find the cdf and pdf for the minimum order statistic 
  $Y_{(1)} = \min_{i \leq n} Y_i$. Hint: $Y_{(1)} > y$
  iff every $Y_i > y$
\end{enumerate}
\begin{enumerate}
  \item We have
  \[
    p_1 = \Prob(Y_i \in [0,1]) = F_Y(1) - F_Y(0) = \left(1 - 2e^{-1}\right) - 0 = 1 - 2e^{-1},
  \]
  \[
    p_2 = \Prob(Y_i \in (1,3]) = F_Y(3) - F_Y(1) = \left(1 - 4e^{-3}\right) - \left(1 - 2e^{-1}\right) = 2e^{-1} - 4e^{-3},
  \]
  \[
    p_3 = \Prob(Y_i \in (3,\infty)) = 1 - F_Y(3) = 4e^{-3}.
  \]
  The mean and variance of each $C_i$ are given by
  \[
    \E(C_i) = n p_i, \quad \Var(C_i) = n p_i (1 - p_i).
  \]
  \item To find the cdf and pdf of the minimum order statistic $Y_{(1)}$, we use the fact that
  \[
    \Prob(Y_{(1)} > y) = \Prob(Y_1 > y, \ldots, Y_n > y) = \left(1 - F_Y(y)\right)^n.
  \]
  Therefore, the cdf of $Y_{(1)}$ is
  \[
    F_{Y_{(1)}}(y) = 1 - \left(1 - F_Y(y)\right)^n,
  \]
  and the pdf is
  \[
    f_{Y_{(1)}}(y) = \frac{d}{dy} F_{Y_{(1)}}(y) = n \left(1 - F_Y(y)\right)^{n-1} f_Y(y).
  \]
\end{enumerate}
\end{problem}

\begin{problem}{3}
  The conditional distribution of $T$ , given $U = u$, has pdf 
  \[
    f_{T|U} (t|u) = \frac{1}{\sqrt{\pi}}u^{1/2} e^{- u t^2}
  \]
for all real $t$, and $U \sim \text{gamma}(3, 1/2)$.
Find the conditional pdf for $U$, given $T = t$, and 
identify the marginal pdf for $T$.
\[
  f_U(u) = \frac{1}{\Gamma(3) (1/2)^3} u^{3-1} e^{-u/(1/2)}
  = \frac{8}{\Gamma(3)} u^{2} e^{-2u} 
  = \frac{8}{2} u^{2} e^{-2u} = 4 u^{2} e^{-2u}
\]
The joint pdf of $(T,U)$ is
\begin{align*}
  f_{T,U}(t,u) &= f_{T|U}(t|u) f_U(u) \\
  &= \frac{1}{\sqrt{\pi}} u^{1/2} e^{-u t^2} \cdot 4 u^{2} e^{-2u} \\
  &= \frac{4}{\sqrt{\pi}} u^{5/2} e^{-u (t^2 + 2)}.
\end{align*}
Thus the conditional pdf of $U$ given $T = t$ is
a gamma distribution with shape parameter $7/2$ and scale parameter $1/(t^2 + 2)$:
\[
  f_{U|T}(u|t) = \frac{(t^2 + 2)^{7/2}}{\Gamma(7/2)} u^{5/2} e^{-u (t^2 + 2)}.
\]
To find the marginal pdf of $T$, we integrate out $U$:
\begin{align*}
  f_T(t) &= \int_0^\infty f_{T,U}(t,u) \, du \\
  &= \int_0^\infty \frac{4}{\sqrt{\pi}} u^{5/2} e^{-u (t^2 + 2)} \, du \\
  &= \frac{4}{\sqrt{\pi}} \cdot \frac{\Gamma(7/2)}{(t^2 + 2)^{7/2}} \\
  &= \frac{4}{\sqrt{\pi}} \cdot \frac{15\sqrt{\pi}/8}{(t^2 + 2)^{7/2}} \\
  &= \frac{15/2}{(t^2 + 2)^{7/2}}.
\end{align*}
This is the pdf of a Student's t-distribution.
\end{problem}

\pagebreak

\begin{problem}{4}
Let $T_1, T_2, \dots$ be iid geometric($p$) random variables, 
and set $\overline{T}_n = \frac{1}{n} \sum_{i=1}^n T_i$.
\begin{enumerate}
  \item Does $- \log(\overline{T}_n)$ converge with probability 1, 
  and, if so, then to what? Explain fully.
  \item Use the delta method to show that $\frac{1}{\overline{T}_n}$
  is asymptotically normal, giving particulars.
\end{enumerate}
\begin{enumerate}
  \item By the Strong Law of Large Numbers (SLLN), we have
  \[
    \overline{T}_n \xrightarrow{w.p. 1} \E(T_1) = \frac{1}{p}.
  \]
  Since the function $g(x) = -\log(x)$ is continuous, by the Continuous Mapping Theorem, we have
  \[
    -\log(\overline{T}_n) \xrightarrow{w.p.1} -\log\left(\frac{1}{p}\right) = \log(p).
  \]
  \item By the Central Limit Theorem (CLT), we have
  \[
    \sqrt{n} \left( \overline{T}_n - \E(T_1) \right) 
    \xrightarrow{D} \text{normal}\left( 0, \Var(T_1) \right).
  \]
  The mean and variance of a geometric($p$) random variable are
  \[
    \E(T_1) = \frac{1}{p}, \quad \Var(T_1) = \frac{1-p}{p^2}.
  \]
  Thus,
  \[
    \sqrt{n} \left( \overline{T}_n - \frac{1}{p} \right) 
    \xrightarrow{D} \text{normal}\left( 0, \frac{1-p}{p^2} \right).
  \]
  Now, we apply the delta method with $g(x) = \frac{1}{x}$, which has derivative
  \[
    g'(x) = -\frac{1}{x^2}.
  \]
  Evaluating at $x = \E(T_1) = \frac{1}{p}$, we have
  \[
    g'\left(\frac{1}{p}\right) = -p^2.
  \]
  Therefore, by the delta method,
  \[
    \sqrt{n} \left( \frac{1}{\overline{T}_n} - p \right) 
    \xrightarrow{D} \text{normal}\left( 0, \left(-p^2\right)^2 \cdot \frac{1-p}{p^2} \right) 
    = \text{normal}\left( 0, p^2 (1-p) \right).
  \]
\end{enumerate}

\end{problem}

\pagebreak

\begin{problem}{5}
Assume $(X, Y)$ is bivariate normal with $\E(X) = \E(Y) = 0$, 
$\Var(X) = \Var(Y) = 1$ and $\text{Corr}(X, Y) = \rho$.
Determine the joint distribution of $(2X + Y, X - 2Y)$ by 
name and parameter values. (There is no need to use pdfs.)
\\\\
Let $Z = (2X + Y, X - 2Y)^T$. Since $(X, Y)$ is bivariate normal,
any linear combination of $X$ and $Y$ is also normally distributed.
Thus, $Z$ is bivariate normal.
Next, we calculate the mean vector and covariance matrix of $Z$.
The mean vector is
\[
  \E(Z) = \begin{pmatrix}
    \E(2X + Y) \\
    \E(X - 2Y)
  \end{pmatrix} = \begin{pmatrix}
    0 \\
    0
  \end{pmatrix}.
\]
The covariance matrix is
\[
  \text{Cov}(Z) = \begin{pmatrix}
    \Var(2X + Y) & \text{Cov}(2X + Y, X - 2Y) \\
    \text{Cov}(X - 2Y, 2X + Y) & \Var(X - 2Y)
  \end{pmatrix}.
\]
Calculating each element, we have
\begin{align*}
  \Var(2X + Y) &= 4\Var(X) + \Var(Y) + 4\text{Cov}(X, Y) = 
  4 + 1 + 4\rho = 5 + 4\rho, \\
  \Var(X - 2Y) &= \Var(X) + 4\Var(Y) - 4\text{Cov}(X, Y) = 
  1 + 4 - 4\rho = 5 - 4\rho, \\
  \text{Cov}(2X + Y, X - 2Y) &= 
  2\Var(X) - 4\text{Cov}(X, Y) + \text{Cov}(Y, X) - 2\Var(Y) \\
  &= 2 - 4\rho + \rho - 2 = -3\rho.
\end{align*}
Thus, the covariance matrix is
\[
  \text{Cov}(Z) = \begin{pmatrix}
    5 + 4\rho & -3\rho \\
    -3\rho & 5 - 4\rho
  \end{pmatrix}.
\]
Therefore, the joint distribution of $(2X + Y, X - 2Y)$ is
bivariate normal with mean vector $\begin{pmatrix} 0 \\ 0 \end{pmatrix}$
and covariance matrix $
\begin{pmatrix} 5 + 4\rho & -3\rho \\ 
  -3\rho & 5 - 4\rho \end{pmatrix}$.

\end{problem}

% --------------------------------------------------------------
%     You don't have to mess with anything below this line.
% --------------------------------------------------------------
\end{document}