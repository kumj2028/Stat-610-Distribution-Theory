%%%%%%%%%%%%%%%%%%%%%%%%%%%%%%%%%%%%%%%%%%%%%%%%%%%%%%%%%%%%%%%
%
% Welcome to writeLaTeX --- just edit your LaTeX on the left,
% and we'll compile it for you on the right. If you give
% someone the link to this page, they can edit at the same
% time. See the help menu above for more info. Enjoy!
%
%%%%%%%%%%%%%%%%%%%%%%%%%%%%%%%%%%%%%%%%%%%%%%%%%%%%%%%%%%%%%%%

% --------------------------------------------------------------
% This is all preamble stuff that you don't have to worry about.
% Head down to where it says "Start here"
% --------------------------------------------------------------
 
\documentclass[12pt]{article}
 
\usepackage[margin=1in]{geometry}
\usepackage{amsmath,amsthm,amssymb}
\usepackage{enumitem}
\usepackage{cancel}

\setlist[enumerate,1]{label={(\alph*)}} %this changes enumerate to (a),(b),...

\usepackage{graphicx} %package to manage images

\newcommand{\A}{{\mathcal{A}}}
\newcommand{\C}{{\mathbb C}}
\newcommand{\CC}{{\mathcal{C}}}
\newcommand{\N}{{\mathbb N}}
\newcommand{\R}{{\mathbb R}}
\newcommand{\Q}{{\mathbb Q}}
\newcommand{\Z}{{\mathbb Z}}

\newcommand{\Aut}{{\rm Aut}}
\newcommand{\End}{{\rm End}}
\newcommand{\Hom}{{\rm Hom}}
\newcommand{\id}{{\rm id}}
\newcommand{\Ima}{{\rm Im}}
\newcommand{\Ker}{{\rm Ker}}
\newcommand{\Mor}{{\rm Mor}}
\newcommand{\Rad}{{\rm Rad}}
\newcommand{\Prob}{{\rm P}}

\renewcommand\labelitemi{-} %this changes itemize bullet points to dashes (-)

\usepackage{listings}
\usepackage{xcolor}

%New colors defined below
\definecolor{codegreen}{rgb}{0,0.6,0}
\definecolor{codegray}{rgb}{0.5,0.5,0.5}
\definecolor{codepurple}{rgb}{0.58,0,0.82}
\definecolor{backcolour}{rgb}{0.95,0.95,0.92}

%Code listing style named "mystyle"
\lstdefinestyle{mystyle}{
  backgroundcolor=\color{backcolour}, commentstyle=\color{codegreen},
  keywordstyle=\color{magenta},
  numberstyle=\tiny\color{codegray},
  stringstyle=\color{codepurple},
  basicstyle=\ttfamily\footnotesize,
  breakatwhitespace=false,         
  breaklines=true,                 
  captionpos=b,                    
  keepspaces=true,                 
  numbers=left,                    
  numbersep=5pt,                  
  showspaces=false,                
  showstringspaces=false,
  showtabs=false,                  
  tabsize=2
}

%"mystyle" code listing set
\lstset{style=mystyle}
 
\newenvironment{theorem}[2][Theorem]{\begin{trivlist}
\item[\hskip \labelsep {\bfseries #1}\hskip \labelsep {\bfseries #2.}]}
{\end{trivlist}}
\newenvironment{lemma}[2][Lemma]{\begin{trivlist}
\item[\hskip \labelsep {\bfseries #1}\hskip \labelsep {\bfseries #2.}]}
{\end{trivlist}}
\newenvironment{exercise}[2][Exercise]{\begin{trivlist}
\item[\hskip \labelsep {\bfseries #1}\hskip \labelsep {\bfseries #2.}]}
{\end{trivlist}}
\newenvironment{problem}[2][Problem]{\begin{trivlist}
\item[\hskip \labelsep {\bfseries #1}\hskip \labelsep {\bfseries #2.}]}
{\end{trivlist}}
\newenvironment{question}[2][Question]{\begin{trivlist}
\item[\hskip \labelsep {\bfseries #1}\hskip \labelsep {\bfseries #2.}]}
{\end{trivlist}}
\newenvironment{corollary}[2][Corollary]{\begin{trivlist}
\item[\hskip \labelsep {\bfseries #1}\hskip \labelsep {\bfseries #2.}]}
{\end{trivlist}}

\newenvironment{solution}{\begin{proof}[Solution]}{\end{proof}}
 
\begin{document}
 
% --------------------------------------------------------------
%                         Start here
% --------------------------------------------------------------
 
\title{Homework 2}%replace X with the appropriate number
\author{Mengxiang Jiang\\ %replace with your name
Stat 610 Distribution Theory} %if necessary, replace with your course title
 
\maketitle
 
\begin{problem}{1} %You can use theorem, exercise, problem, or question here.
  \textit{Statistical Inference} by Casella and Berger, 2nd Edition, Chapter 1, 
  Exercise 34.
  \begin{itemize}
    \item[34.] Two litters of a particular rodent species have been born,
    one with two brown-haired and one gray-haired (litter 1), and the other
    with three brown-haired and two gray-haired (litter 2). We select a
    litter at random and then select an offspring at random from the selected
    litter.
    \begin{enumerate}
      \item What is the probability that the animal chosen is brown-haired?
      \item Given that a brown-haired offspring was selected, what is the
      probability that the sampling was from litter 1?
    \end{enumerate}
    \begin{enumerate}
      \item Let $A$ be the event that the animal chosen is brown-haired,
      Let $B_i$ be the event that the sampling was from litter $i$, $i=1,2$.
      Then we have 
      \[
        \Prob(A) = \Prob(A|B_1)\Prob(B_1) + \Prob(A|B_2)\Prob(B_2) 
        = \frac{2}{3}\cdot\frac{1}{2} + \frac{3}{5}\cdot\frac{1}{2} 
        = \frac{19}{30}.
      \]
      \item By Bayes' theorem, we have
      \[
        \Prob(B_1|A) = \frac{\Prob(A|B_1)\Prob(B_1)}{\Prob(A)} 
        = \frac{\frac{2}{3}\cdot\frac{1}{2}}{\frac{19}{30}} 
        = \frac{5}{19}.
      \]
    \end{enumerate}
  \end{itemize}
\end{problem}

\begin{problem}{2}
  \begin{enumerate}
    \item Prove this alternative to Bayes' rule:
    \[
      \log\left(\frac{\Prob(A|B)}{\Prob(A^c|B)}\right)
      = \log\left(\frac{\Prob(A)}{\Prob(A^c)}\right)
      + \log\left(\frac{\Prob(B|A)}{\Prob(B|A^c)}\right).
    \]
    This expression is useful in genetics: conditional log odds of disease
    $(A)$ given gene $(B)$ = unconditional log odds of disease + log ratio
    of gene prevalence.
    \item Suppose $B_1,\dots,B_n$ are disjoint. Show that
    \[
      \Prob\left(B_j|\bigcup_{i=1}^n B_i\right) = \frac{\Prob(B_j)}
      {\sum_{i=1}^n \Prob(B_i)}. \quad \text{for each } j \in \{1,\dots,n\}.
    \]
  \end{enumerate}
  \begin{enumerate}
    \item By definition, we have
    \[
      \log\left(\frac{\Prob(A|B)}{\Prob(A^c|B)}\right)
      = \log\left(\frac{\frac{\Prob(A\cap B)}{\Prob(B)}}
      {\frac{\Prob(A^c\cap B)}{\Prob(B)}}\right)
      = \log\left(\frac{\Prob(A\cap B)}{\Prob(A^c\cap B)}\right).
    \]
    By the definition of conditional probability again and
    $\log(AB) = \log(A)+\log(B)$, we have
    \[
      \log\left(\frac{\Prob(A\cap B)}{\Prob(A^c\cap B)}\right)
      = \log\left(\frac{\Prob(A)\Prob(B|A)}{\Prob(A^c)\Prob(B|A^c)}\right)
      = \log\left(\frac{\Prob(A)}{\Prob(A^c)}\right)
      + \log\left(\frac{\Prob(B|A)}{\Prob(B|A^c)}\right).
    \]
  \end{enumerate}
\end{problem}

% --------------------------------------------------------------
%     You don't have to mess with anything below this line.
% --------------------------------------------------------------
\end{document}