%%%%%%%%%%%%%%%%%%%%%%%%%%%%%%%%%%%%%%%%%%%%%%%%%%%%%%%%%%%%%%%
%
% Welcome to writeLaTeX --- just edit your LaTeX on the left,
% and we'll compile it for you on the right. If you give
% someone the link to this page, they can edit at the same
% time. See the help menu above for more info. Enjoy!
%
%%%%%%%%%%%%%%%%%%%%%%%%%%%%%%%%%%%%%%%%%%%%%%%%%%%%%%%%%%%%%%%

% --------------------------------------------------------------
% This is all preamble stuff that you don't have to worry about.
% Head down to where it says "Start here"
% --------------------------------------------------------------
 
\documentclass[12pt]{article}
 
\usepackage[margin=1in]{geometry}
\usepackage{amsmath,amsthm,amssymb}
\usepackage{enumitem}
\usepackage{cancel}

\setlist[enumerate,1]{label={(\alph*)}} %this changes enumerate to (a),(b),...

\usepackage{graphicx} %package to manage images

\newcommand{\A}{{\mathcal{A}}}
\newcommand{\C}{{\mathbb C}}
\newcommand{\CC}{{\mathcal{C}}}
\newcommand{\N}{{\mathbb N}}
\newcommand{\R}{{\mathbb R}}
\newcommand{\Q}{{\mathbb Q}}
\newcommand{\Z}{{\mathbb Z}}

\newcommand{\Aut}{{\rm Aut}}
\newcommand{\End}{{\rm End}}
\newcommand{\Hom}{{\rm Hom}}
\newcommand{\id}{{\rm id}}
\newcommand{\Ima}{{\rm Im}}
\newcommand{\Ker}{{\rm Ker}}
\newcommand{\Mor}{{\rm Mor}}
\newcommand{\Rad}{{\rm Rad}}
\newcommand{\Prob}{{\sf P}}
\newcommand{\E}{{\sf E}}
\newcommand{\Var}{{\sf Var}}
\newcommand{\Cov}{{\sf Cov}}

\renewcommand\labelitemi{-} %this changes itemize bullet points to dashes (-)

\usepackage{listings}
\usepackage{xcolor}

%New colors defined below
\definecolor{codegreen}{rgb}{0,0.6,0}
\definecolor{codegray}{rgb}{0.5,0.5,0.5}
\definecolor{codepurple}{rgb}{0.58,0,0.82}
\definecolor{backcolour}{rgb}{0.95,0.95,0.92}

%Code listing style named "mystyle"
\lstdefinestyle{mystyle}{
  backgroundcolor=\color{backcolour}, commentstyle=\color{codegreen},
  keywordstyle=\color{magenta},
  numberstyle=\tiny\color{codegray},
  stringstyle=\color{codepurple},
  basicstyle=\ttfamily\footnotesize,
  breakatwhitespace=false,         
  breaklines=true,                 
  captionpos=b,                    
  keepspaces=true,                 
  numbers=left,                    
  numbersep=5pt,                  
  showspaces=false,                
  showstringspaces=false,
  showtabs=false,                  
  tabsize=2
}

%"mystyle" code listing set
\lstset{style=mystyle}
 
\newenvironment{theorem}[2][Theorem]{\begin{trivlist}
\item[\hskip \labelsep {\bfseries #1}\hskip \labelsep {\bfseries #2.}]}
{\end{trivlist}}
\newenvironment{lemma}[2][Lemma]{\begin{trivlist}
\item[\hskip \labelsep {\bfseries #1}\hskip \labelsep {\bfseries #2.}]}
{\end{trivlist}}
\newenvironment{exercise}[2][Exercise]{\begin{trivlist}
\item[\hskip \labelsep {\bfseries #1}\hskip \labelsep {\bfseries #2.}]}
{\end{trivlist}}
\newenvironment{problem}[2][Problem]{\begin{trivlist}
\item[\hskip \labelsep {\bfseries #1}\hskip \labelsep {\bfseries #2.}]}
{\end{trivlist}}
\newenvironment{question}[2][Question]{\begin{trivlist}
\item[\hskip \labelsep {\bfseries #1}\hskip \labelsep {\bfseries #2.}]}
{\end{trivlist}}
\newenvironment{corollary}[2][Corollary]{\begin{trivlist}
\item[\hskip \labelsep {\bfseries #1}\hskip \labelsep {\bfseries #2.}]}
{\end{trivlist}}

\newenvironment{solution}{\begin{proof}[Solution]}{\end{proof}}
 
\begin{document}
 
% --------------------------------------------------------------
%                         Start here
% --------------------------------------------------------------
 
\title{Exam 2 2023}%replace X with the appropriate number
\author{Mengxiang Jiang\\ %replace with your name
Stat 610 Distribution Theory} %if necessary, replace with your course title
 
\maketitle
 
\begin{problem}{1} %You can use theorem, exercise, problem, or question here.
  Suppose $T$ has pdf $f_T (t) = \frac{1}{t \log(2)}$ for $1 \leq t \leq 2$ 
  and the conditional distribution of
  $X$, given $T$, is exponential with scale parameter $T$. That is, 
  $f_{X|T} (x|t) = \frac{1}{t} e^{-x/t}$ for $x > 0$.
  \begin{enumerate}
    \item Express the marginal pdf for $X$ as a simple integral and then 
    attempt the integration. It will help to note that the anti-derivative of 
    $\frac{a}{t^2} e^{-a/t}$ is $e^{-a/t}$.
    \item Find $\E(X)$ by using iterated expectation.
  \end{enumerate}
  \begin{enumerate}
    \item The marginal pdf for $X$ is
    \begin{align*}
      f_X (x) 
      &= \int_{1}^{2} f_{X|T} (x|t) f_T (t) dt \\
      &= \int_{1}^{2} \frac{1}{t} e^{-x/t} \cdot \frac{1}{t \log(2)} dt \\
      &= \frac{1}{\log(2)} \int_{1}^{2} \frac{1}{t^2} e^{-x/t} dt \\
      &= \frac{1}{\log(2)} \left[\frac{1}{x} e^{-x/t} \right]_{t=1}^{t=2} \\
      &= \frac{1}{x \log(2)} \left(e^{-x/2} - e^{-x}\right), \quad x > 0.
    \end{align*}
    \item By iterated expectation, we have
    \begin{align*}
      \E(X) 
      &= \E\left(\E(X|T)\right) \\
      &= \E(T) \\
      &= \int_{1}^{2} t \cdot \frac{1}{t \log(2)} dt \\
      &= \frac{1}{\log(2)} \int_{1}^{2} 1 dt \\
      &= \frac{1}{\log(2)}.
    \end{align*}
  \end{enumerate}
\end{problem}

\begin{problem}{2} 
  Let $T \sim \text{gamma}(\alpha, 1)$ and $R \sim \text{uniform}(0, 1)$, 
  independent. Let $(X, Y) = (RT, (1 - R)T)$.
  \begin{enumerate}
    \item Compute $\E(XY)$. Hint: it may help to think about this in 
    terms of expectations for $(R, T)$.
    \item Find the joint pdf for $(X, Y)$. (Note that $T = X + Y$ and 
    $R = \frac{X}{X + Y}$.)
  \end{enumerate}
  \begin{enumerate}
    \item 
    \begin{align*}
      \E(XY) 
      &= \E\left((RT)((1-R)T)\right) \\
      &= \E\left(T^2 R(1 - R)\right) \\
      &= \E\left(T^2\right) \E(R(1 - R)) \\
      &= (\text{Var}(T) + [\E(T)]^2) \left(\E(R) - \E(R^2)\right) \\
      &= (\alpha + \alpha^2) \left(\frac{1}{2} - \frac{1}{3}\right) \\
      &= (\alpha + \alpha^2) \cdot \frac{1}{6} \\
      &= \frac{\alpha(\alpha + 1)}{6}.
    \end{align*}
    \item The inverse transformation is
    \[
      r = \frac{x}{x + y}, \quad t = x + y. 
    \]
    The Jacobian determinant is
    \[
      J =  \begin{vmatrix}
        \frac{\partial r}{\partial x} & \frac{\partial r}{\partial y} \\
        \frac{\partial t}{\partial x} & \frac{\partial t}{\partial y}
      \end{vmatrix} 
      = \begin{vmatrix}
        \frac{y}{(x + y)^2} & -\frac{x}{(x + y)^2} \\
        1 & 1
      \end{vmatrix}
      = \frac{1}{x + y}.
    \]
    The joint pdf for $(X, Y)$ is
    \begin{align*}
      f_{X, Y} (x, y) 
      &= f_{R, T} \left(\frac{x}{x + y}, x + y\right) 
      \left| J \right| \\
      &= f_R \left(\frac{x}{x + y}\right) f_T (x + y) \cdot \frac{1}{x + y} \\
      &= 1 \cdot \frac{1}{\Gamma(\alpha)} (x + y)^{\alpha - 1} e^{-(x + y)} 
      \cdot \frac{1}{x + y} \\
      &= \frac{1}{\Gamma(\alpha)} (x + y)^{\alpha - 2} e^{-(x + y)}, 
      \quad x > 0, y > 0.
    \end{align*}
  \end{enumerate}
\end{problem}

\begin{problem}{3}
  Recall that $\Cov(X, Y) = \E((X - \mu_X)(Y - \mu_Y))$. Prove, for any 
  real-valued $a$, that
  \[
    \E((X - \mu_X)(Y - \mu_Y)) = \E((X - a)(Y - \mu_Y)).
  \]
  \textit{Do not assume or use properties about covariances; 
  just use expectations.}
  \\\\
  \begin{align*}
    \E((X - a)(Y - \mu_Y)) 
    &= \E((X - \mu_X + \mu_X - a)(Y - \mu_Y)) \\
    &= \E((X - \mu_X)(Y - \mu_Y)) + \E((\mu_X - a)(Y - \mu_Y)) \\
    &= \E((X - \mu_X)(Y - \mu_Y)) + (\mu_X - a) \E(Y - \mu_Y) \\
    &= \E((X - \mu_X)(Y - \mu_Y)) + (\mu_X - a) ( \mu_Y - \mu_Y) \\
    &= \E((X - \mu_X)(Y - \mu_Y)) + 0 \\
    &= \E((X - \mu_X)(Y - \mu_Y)).
  \end{align*}
\end{problem}

\begin{problem}{4}
  Suppose $\beta > 0$ and $W$ has probability mass function
  \[
    f_W (w) = c(\beta) w^2 e^{-(w/\beta)^2}
  \]
  for $w = 1, 2, 3, \ldots$,
  and $c(\beta)$ is defined to be the constant so that $f_W (w)$ sums to 1. 
  Does this model define an exponential family with parameter $\beta$? 
  Why or why not?
  \\\\
  An exponential family can be written in the form
  \[
    f(w) = c(\beta) h(w) e^{g(\beta) t(w)},
  \]
  for functions $c, h, g, t$.
  We can rewrite this pmf as
  \begin{align*}
    f_W (w) 
    &= c(\beta) w^2 e^{-(w/\beta)^2} \\
    &= c(\beta) w^2 e^{-w^2 / \beta^2} \\
    &= c(\beta) h(w) e^{g(\beta) t(w)},
    \quad \text{where } h(w) = w^2, \quad g(\beta) = -\frac{1}{\beta^2}, 
    \quad t(w) = w^2.
  \end{align*}
  Thus, this model defines an exponential family with parameter $\beta$.
\end{problem}

% --------------------------------------------------------------
%     You don't have to mess with anything below this line.
% --------------------------------------------------------------
\end{document}