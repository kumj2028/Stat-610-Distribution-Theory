%%%%%%%%%%%%%%%%%%%%%%%%%%%%%%%%%%%%%%%%%%%%%%%%%%%%%%%%%%%%%%%
%
% Welcome to writeLaTeX --- just edit your LaTeX on the left,
% and we'll compile it for you on the right. If you give
% someone the link to this page, they can edit at the same
% time. See the help menu above for more info. Enjoy!
%
%%%%%%%%%%%%%%%%%%%%%%%%%%%%%%%%%%%%%%%%%%%%%%%%%%%%%%%%%%%%%%%

% --------------------------------------------------------------
% This is all preamble stuff that you don't have to worry about.
% Head down to where it says "Start here"
% --------------------------------------------------------------
 
\documentclass[12pt]{article}
 
\usepackage[margin=1in]{geometry}
\usepackage{amsmath,amsthm,amssymb}
\usepackage{enumitem}
\usepackage{cancel}

\setlist[enumerate,1]{label={(\alph*)}} %this changes enumerate to (a),(b),...

\usepackage{graphicx} %package to manage images

\newcommand{\A}{{\mathcal{A}}}
\newcommand{\C}{{\mathbb C}}
\newcommand{\CC}{{\mathcal{C}}}
\newcommand{\N}{{\mathbb N}}
\newcommand{\R}{{\mathbb R}}
\newcommand{\Q}{{\mathbb Q}}
\newcommand{\Z}{{\mathbb Z}}

\newcommand{\Aut}{{\rm Aut}}
\newcommand{\End}{{\rm End}}
\newcommand{\Hom}{{\rm Hom}}
\newcommand{\id}{{\rm id}}
\newcommand{\Ima}{{\rm Im}}
\newcommand{\Ker}{{\rm Ker}}
\newcommand{\Mor}{{\rm Mor}}
\newcommand{\Rad}{{\rm Rad}}
\newcommand{\Prob}{{\sf P}}
\newcommand{\E}{{\sf E}}
\newcommand{\Var}{{\sf Var}}

\renewcommand\labelitemi{-} %this changes itemize bullet points to dashes (-)

\usepackage{listings}
\usepackage{xcolor}

%New colors defined below
\definecolor{codegreen}{rgb}{0,0.6,0}
\definecolor{codegray}{rgb}{0.5,0.5,0.5}
\definecolor{codepurple}{rgb}{0.58,0,0.82}
\definecolor{backcolour}{rgb}{0.95,0.95,0.92}

%Code listing style named "mystyle"
\lstdefinestyle{mystyle}{
  backgroundcolor=\color{backcolour}, commentstyle=\color{codegreen},
  keywordstyle=\color{magenta},
  numberstyle=\tiny\color{codegray},
  stringstyle=\color{codepurple},
  basicstyle=\ttfamily\footnotesize,
  breakatwhitespace=false,         
  breaklines=true,                 
  captionpos=b,                    
  keepspaces=true,                 
  numbers=left,                    
  numbersep=5pt,                  
  showspaces=false,                
  showstringspaces=false,
  showtabs=false,                  
  tabsize=2
}

%"mystyle" code listing set
\lstset{style=mystyle}
 
\newenvironment{theorem}[2][Theorem]{\begin{trivlist}
\item[\hskip \labelsep {\bfseries #1}\hskip \labelsep {\bfseries #2.}]}
{\end{trivlist}}
\newenvironment{lemma}[2][Lemma]{\begin{trivlist}
\item[\hskip \labelsep {\bfseries #1}\hskip \labelsep {\bfseries #2.}]}
{\end{trivlist}}
\newenvironment{exercise}[2][Exercise]{\begin{trivlist}
\item[\hskip \labelsep {\bfseries #1}\hskip \labelsep {\bfseries #2.}]}
{\end{trivlist}}
\newenvironment{problem}[2][Problem]{\begin{trivlist}
\item[\hskip \labelsep {\bfseries #1}\hskip \labelsep {\bfseries #2.}]}
{\end{trivlist}}
\newenvironment{question}[2][Question]{\begin{trivlist}
\item[\hskip \labelsep {\bfseries #1}\hskip \labelsep {\bfseries #2.}]}
{\end{trivlist}}
\newenvironment{corollary}[2][Corollary]{\begin{trivlist}
\item[\hskip \labelsep {\bfseries #1}\hskip \labelsep {\bfseries #2.}]}
{\end{trivlist}}

\newenvironment{solution}{\begin{proof}[Solution]}{\end{proof}}
 
\begin{document}
 
% --------------------------------------------------------------
%                         Start here
% --------------------------------------------------------------
 
\title{Exam 1 2023}%replace X with the appropriate number
\author{Mengxiang Jiang\\ %replace with your name
Stat 610 Distribution Theory} %if necessary, replace with your course title
 
\maketitle
 
\begin{problem}{1} %You can use theorem, exercise, problem, or question here.
  The mean and variance of $X \sim \text{binomial}(n, p)$ are $np$ 
  and $np(1 - p)$, respectively. What is $\E(X(X + 1))$? Try to simplify.
  \\\\
  \[
    \begin{aligned}
      \E(X(X + 1)) &= \E(X^2 + X) \\
      &= \E(X^2) + \E(X) \\
      &= \Var(X) + [\E(X)]^2 + \E(X) \\
      &= np(1 - p) + (np)^2 + np \\
      &= n^2p^2 + np.
    \end{aligned}
  \]
\end{problem}

\begin{problem}{2}
  Let $Y \sim \text{uniform}(1, 3)$.
  \begin{enumerate}
    \item Find the pdf for $V = \sqrt{Y}$.
    \item Find $\E(V)$ and $v_{0.50}$ (the $0.50$-quantile for $V$). 
    Are they the same value?
  \end{enumerate} 
  \begin{enumerate}
    \item \[
      f_Y (y) =
      \begin{cases}
        \frac{1}{3 - 1} = \frac{1}{2} & \text{if } 1 < y < 3, \\
        0 & \text{otherwise}.
      \end{cases}
    \]
    Since $V = \sqrt{Y}$, we have $Y = V^2$ and $\frac{dY}{dV} = 2V$.
    By the change of variables formula, we have
    \[
      f_V (v) = f_Y (v^2) \left| \frac{dY}{dV} \right|
      = f_Y (v^2) \cdot 2v.
    \]
    Note that $1 < y < 3$ is equivalent to $1 < v^2 < 3$, or $1 < v < \sqrt{3}$.
    Thus,
    \[
      f_V (v) =
      \begin{cases}
        \frac{1}{2} \cdot 2v = v & \text{if } 1 < v < \sqrt{3}, \\
        0 & \text{otherwise}.
      \end{cases}
    \]
    \item \[
      \begin{aligned}
        \E(V) &= \int_{-\infty}^{\infty} v f_V (v) dv \\
        &= \int_1^{\sqrt{3}} v \cdot v dv \\
        &= \int_1^{\sqrt{3}} v^2 dv \\
        &= \left. \frac{v^3}{3} \right|_1^{\sqrt{3}} \\
        &= \frac{(\sqrt{3})^3}{3} - \frac{1^3}{3} = \frac{3\sqrt{3}-1}{3}.
      \end{aligned}
    \]
    To find $v_{0.50}$, we first find the cdf of $V$.
    \[
      F_V (v) =
      \begin{cases}
        0 & \text{if } v \le 1, \\
        \int_1^v t dt = \left. \frac{t^2}{2} \right|_1^v = \frac{v^2}{2} - \frac{1}{2} 
        = \frac{v^2 - 1}{2} & \text{if } 1 < v < \sqrt{3}, \\
        1 & \text{if } v \ge \sqrt{3}.
      \end{cases}
    \]
    The $0.50$-quantile must satisfy the equation
    \[
      F_V (v_{0.50}) = 0.50.
    \]
    Thus,
    \[
      \frac{v_{0.50}^2 - 1}{2} = 0.50 \implies v_{0.50}^2 - 1 = 1 \implies v_{0.50}^2 = 2
      \implies v_{0.50} = \sqrt{2}.
    \]
    Note that $\E(V) = \frac{3\sqrt{3}-1}{3} \approx 1.40$ 
    and $v_{0.50} = \sqrt{2} \approx 1.41$.
    They are not the same value.
  \end{enumerate}

\end{problem}

\begin{problem}{3}
  80\% of connections to a certain server are of Type A 
  and the other 20\% are of Type B. Let
  $T$ be the lifetime of a connection.\\
  For Type A, $T \sim \text{exponential}(1)$: 
  $\Prob(T \le t| \text{Type A}) = 1 - e^{-t}$.\\
  For Type B, $T \sim \text{gamma}(2, .5)$: 
  $\Prob(T \le t| \text{Type B}) = 1 - (1 + 2t)e^{-2t}$.\\
  \begin{enumerate}
    \item Provide the cdf and pdf for a randomly chosen connection. 
    That is, give the unconditional $\Prob(T \le t)$ and then determine the pdf.
    \item The moment generating function for $T$ is 
    $M_T (s) = \frac{0.8}{1-s} + \frac{0.2}{(1-0.5s)^{2}}$ for $s < 1$.
    Use this to find $\E(T)$.
  \end{enumerate}
  \begin{enumerate}
    \item By the law of total probability, we have
    \[
      \Prob(T \le t) = \Prob(T \le t | \text{Type A}) \Prob(\text{Type A})
      + \Prob(T \le t | \text{Type B}) \Prob(\text{Type B}).
    \]
    \[
      \begin{aligned}
        \Prob(T \le t) &= (1 - e^{-t}) \cdot 0.8 + [1 - (1 + 2t)e^{-2t}] \cdot 0.2 \\
        &= 0.8 - 0.8e^{-t} + 0.2 - 0.2(1 + 2t)e^{-2t} \\
        &= 1 - 0.8e^{-t} - 0.2(1 + 2t)e^{-2t},
      \end{aligned}
    \]
    Since this is a cdf, the pdf can be found by differentiating the cdf:
    \[
      \begin{aligned}
        f_T (t) &= \frac{d}{dt} \Prob(T \le t) \\
        &= \frac{d}{dt} \left[ 1 - 0.8e^{-t} - 0.2(1 + 2t)e^{-2t} \right] \\
        &= 0 + 0.8e^{-t} - 0.2 \frac{d}{dt} \left[ (1 + 2t)e^{-2t} \right] \\
        &= 0.8e^{-t} - 0.2 \left[ (1 + 2t)(-2e^{-2t}) + 2e^{-2t} \right] 
        \quad (\text{by product rule})\\
        &= 0.8e^{-t} - 0.2 \left[ -2(1 + 2t)e^{-2t} + 2e^{-2t} \right] \\
        &= 0.8e^{-t} - 0.2 \left[ (-2 - 4t + 2)e^{-2t} \right] \\
        &= 0.8e^{-t} - 0.2(-4te^{-2t}) = 0.8e^{-t} + 0.8te^{-2t}.
      \end{aligned}
    \]
    \item Using the given mgf, we can find $\E(T)$ as follows.
    \[
      \begin{aligned}
        \E(T) &= M_T' (0) \\
        &= \frac{d}{ds} \left[ \frac{0.8}{1-s} + 
        \frac{0.2}{(1-0.5s)^{2}} \right]_{s=0} \\
        &= 0.8 \frac{d}{ds} (1-s)^{-1} + 
        0.2 \frac{d}{ds} (1-0.5s)^{-2} \bigg|_{s=0} \\
        &= 0.8 (-(1-s)^{-2})(-1) + 0.2 (-2)(1-0.5s)^{-3}(-0.5) \bigg|_{s=0} \\
        &= 0.8 (1-s)^{-2} + 0.2 (1-0.5s)^{-3} \bigg|_{s=0} \\
        &= 0.8 (1-0)^{-2} + 0.2 (1-0)^{-3} = 0.8 + 0.2 = 1.
      \end{aligned}
    \]
    If we use the pdf found in part (a), we can also find $\E(T)$ as follows.
    \[
      \begin{aligned}
        \E(T) &= \int_{-\infty}^{\infty} t f_T (t) dt \\
        &= \int_0^{\infty} t (0.8e^{-t} + 0.8te^{-2t}) dt \\
        &= 0.8 \int_0^{\infty} t e^{-t} dt + 0.8 \int_0^{\infty} t^2 e^{-2t} dt \\
        &= 0.8 \cdot 1 + 0.8 \cdot \frac{1}{4} \int_0^{\infty} (2t)^2 e^{-2t} dt 
        \quad (\text{using } \int_0^{\infty} y^m e^{-y} dy = m!) \\
        &= 0.8 + 0.8 \cdot \frac{1}{4} \int_0^{\infty} y^2 e^{-y} dy
        \quad (\text{using } y = 2t) \\
        &= 0.8 + 0.8 \cdot \frac{1}{4} \cdot 2!
        \quad (\text{using } \int_0^{\infty} y^m e^{-y} dy = m!) \\
        &= 0.8 + 0.8 \cdot \frac{2}{8} = 0.8 + 0.2 = 1.
      \end{aligned}
    \]
  \end{enumerate}
\end{problem}

\begin{problem}{4}
  For a giveaway promotional event, a store randomly selects 40 
  pairs of shoes from its stock of 1000 pairs. The stock consists of 
  400 Hoka pairs, 350 New Balance pairs and 250 Saucony pairs. 
  Give expressions for (a) the chance that $x$ Hoka pairs, 
  $y$ New Balance pairs and $z$ Saucony pairs are selected and for 
  (b) the chance at least $w$ Hoka or New Balance pairs
  are selected. Be complete.

  \begin{enumerate}
    \item By the multivariate hypergeometric distribution, we have
    \[
      \Prob(X = x, Y = y, Z = z) = \frac{\binom{400}{x} \binom{350}{y} \binom{250}{z}}
      {\binom{1000}{40}},
    \]
    where $X$, $Y$ and $Z$ are the number of Hoka, New Balance and Saucony pairs
    selected, respectively, and $x$, $y$ and $z$ are nonnegative integers such that
    $x + y + z = 40$, $x \le 400$, $y \le 350$ and $z \le 250$.
    \item Let $A$ be the event that at least $w$ Hoka or New Balance pairs
    are selected. Then $A = B \cup C$, where $B$ is the event that at least $w$ Hoka pairs
    are selected and $C$ is the event that at least $w$ New Balance pairs are selected.
    By the inclusion-exclusion principle, we have
    \[
      \Prob(A) = \Prob(B) + \Prob(C) - \Prob(B \cap C).
    \]
    We can calculate $\Prob(B)$, $\Prob(C)$ and $\Prob(B \cap C)$ as follows.
    \[
      \Prob(B) = \sum_{x=w}^{40} \sum_{y=0}^{40-x} 
      \frac{\binom{400}{x} \binom{350}{y} \binom{250}{40-x-y}}{\binom{1000}{40}},
    \]
    \[
      \Prob(C) = \sum_{y=w}^{40} \sum_{x=0}^{40-y} 
      \frac{\binom{400}{x} \binom{350}{y} \binom{250}{40-x-y}}{\binom{1000}{40}},
    \]
    \[
      \Prob(B \cap C) = \sum_{x=w}^{40} \sum_{y=w}^{40-x} 
      \frac{\binom{400}{x} \binom{350}{y} \binom{250}{40-x-y}}{\binom{1000}{40}}.
    \]
    Thus,
    \[
      \begin{aligned}
        \Prob(A) &= \Prob(B) + \Prob(C) - \Prob(B \cap C) \\
        &= \sum_{x=w}^{40} \sum_{y=0}^{40-x} 
        \frac{\binom{400}{x} \binom{350}{y} \binom{250}{40-x-y}}{\binom{1000}{40}} \\
        &\quad + \sum_{y=w}^{40} \sum_{x=0}^{40-y} 
        \frac{\binom{400}{x} \binom{350}{y} \binom{250}{40-x-y}}{\binom{1000}{40}} \\
        &\quad - \sum_{x=w}^{40} \sum_{y=w}^{40-x} 
        \frac{\binom{400}{x} \binom{350}{y} \binom{250}{40-x-y}}{\binom{1000}{40}}.
      \end{aligned}
    \]
  \end{enumerate}
\end{problem}

% --------------------------------------------------------------
%     You don't have to mess with anything below this line.
% --------------------------------------------------------------
\end{document}